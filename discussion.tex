\section*{Discussion}
%Future prspectives: Calibration? integration of gene expression? non-coding? positive control? Copy number variants ? 

%Despite the its decreasing costs, whole-genome sequencing is not yet applied to the diagnosis of aneuploidies  ... \\
%Rare variants have large effects, natural selection prevent them to become common 
%We developed a pipeline to select cases of PLs in which the genome of the PoC is euploid and the mother does not present obvious comorbidities. These cases are similar to cases of idiopathic miscarriages that can be used to target the identification of small-size lethal genomic variants through whole genome sequencing.\\

%The identification of small variants requires large sample size. We observe the fraction of samples which... therefore we estimate that the number of samples to collect shuold be  X times the number of samples to be sequenced ...  a sample size of ...  is required to .... Figure \ref{fig:fractions}\\

%We also learned something about miscarriages: report aggregate statistics of qfPCR and arrayCGH when will be available.\\ 

%samples not used for sequencing can be used to study chromosomal rearrangements 

%Limitations: \\
%- array CGH: oinversion not visible.  only deletion and duplication but when complex it is impossible to determine the  order of the fragments. Complex chromosomal rearrangements  and Chromoanagenesis that do not involve copy nuber variants can not be identified.\\ 
%- Is it valid price-wise or better do low-coverage sequencing? 
