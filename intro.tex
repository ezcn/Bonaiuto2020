\section*{Introduction}
\gp 
%Human reproduction is highly ineffective and it is estimated that 10-20\% of all pregnancies end in early miscarriage or early pregnancy loss (PL) during the first trimester [REF SILVIA] and up to 50\% of cases of RPL do not have a clearly defined etiology\cite{practice2012evaluation}. Miscarriage is defined as the death of the fetus within 20-28 week of gestation\cite{pmid27994187, pmid25055407, pmid25944391, pmid11821293, stephenson2002cytogenetic, pmid25681385, pmid29858908}. Recurrent pregnancy loss (RPL) is defined as the loss of two or more consecutive pregnancies\cite{green2019review}. RPL has a high impact on public health since it affects up to 3-5\% of couples[REF PER QUESTO]. For 50-60\% of RPL cases the cause are known to be structural genetic, endocrine, anatomic, thrombophilic, autoimmune, and environmental factors, however for the remaining 40-50\% the causes are unknown\cite{pmid27994187, pmid25944391, pmid11821293, stephenson2002cytogenetic, pmid22796359, pmid22672580, pmid25681385, gaboon2013recurrent, pmid30642578}.\\


%The diagnosis of miscarriages is based on embryo heart activity and gestational sac features revealed by ultrasonography\cite{doubilet2013diagnostic}. Nevertheless, diagnosis take place only after the death of the embryo, and only few cases there are followed-up to understand the genetic causes with techniques that can discriminate anouplidies (kariotyping, quantitative PCR) or large deleterious copy number variants (comparative genomic hybridization), while no information is available on small-size DNA changes incompatible with life. Therefore, the current ability of inform prognosis and manage decision in cases of perinatal lethality is limited, with important consequences in counseling for RPLs and \textit{in-vitro} fertilization.\\

%Among PL caused by genetic defects, it is estimated that 50-70\% [check PERCENT] is due to meiotic chromosome segregation errors, whose frquency increases with increasing maternal age\cite{pmid30393965, pmid10864550, pmid20041396}. Sperm DNA fragmentation caused by oxidative stress also causes PL through impairment of placentation \cite{pmid2972074, pmid30448091, pmid30602480}. %as well as increased mutation rate in sperm that increases with male age\cite{kong2012rate}. 
%To date, PLs are studied using parental genetic information \cite{laisk2019genetic, pereza2017systematic}. Only a few studies have focused on deep understanding of the genetic causes through the analysis of the fetal genome sequence\cite{filges2015exome}[ADD REF]. However, these few cases never considered whole genome sequencing but rather concentrated on restricted target regions related to specific medical cases. Therefore, very little is known about the genetic mutations that effectively cause the death of the embryo and there is the need for large scale projects that systematically target small-size genetic mutations. % to help understanding unexplained PLs. 


%In this study we develop a pipeline for selecting cases of idiopathic PL to be studied through whole-genome sequencing of DNA from product of conception (PoC).  \\
%We find that... \\
%Our study will facilitate the development of a larger-scale project for developing molecular diagnosis of PL. 


%The aims of this study is to identify genetic variants to cause miscarriages and improve prenatal diagnosis. We analyze DNA obtained from tissue of chorionic villi from cases of SA and RM, comparing with controls, voluntary termination of pregnancy (VTP)[\cite{pmid28352815}]. In the first phase, the data on habits and health status of women subjected to abortion were analyzed, with particular attention to gestational age, BMI, menarche age, age of women at time of the pregnancy, use of alcohol, drugs or coffeine etc. An initial stratification of the samples was performed based on the results of these data. Subsequently the samples were analyzed by qfPCR, in order to remove from the miscarriage study, the samples with chromosomal anomalies. In the second phase a CGH array was performed to exclude anomalies not visible to the qfPCR. In the third phase, the fetal DNA sequences that were found to be suitable, have been analyzed.



 

%The protocols provide a dedicated and focused service to couples who already had experience of  RPL, so it doesn’t prevent from a new event of miscarriage. The goal of the pregnancy loss is have a prevention center, as write in study of Bruce K.Young, 2019. An essential part of the management of couples with recurrent miscarriage is to give trustworthy advice on the prognosis for the next pregnancy for the couples to be able to decide for or against further pregnancy attempts. In the field of recurrent miscarriage, however, a distinct problem is the lack of comparability between estimates of the chance of subsequent successful pregnancy outcomes reported in various studies. The chance of live birth in the next pregnancy in women with three, four, and five previous miscarriages has been reported variably to be between 63 and 87\%, 44 and 73\%, and 25 and 52\%, respectively. Nowadays, the first visit after referral for RPL should allow time for the clinician to review the patient’s history, which includes medical, obstetric, and family history, but also information on lifestyle of both the male and female partner. Studies have suggested an impact of the following lifestyle factors on the risk of RPL: smoking, excessive alcohol consumption, excessive exercise and being overweight or underweight. In addition to lifestyle factors, information should be collected on a previous diagnosis of medical conditions that may be associated with RPL, including thrombophilia, PCOS, and diabetes, or a family history of hereditary thrombophilia. Medical and family history could be helpful in deciding which investigations are relevant for the individual patient (age, fertility/sub-fertility, pregnancy history, family history, previous investigations and/or treatments). However, no studies have been performed that could advise clinicians on which diagnostic tests are relevant for a specific patient and, more importantly, which are not [\cite{pmid29858908, pmid22183209, pmid22835448, pmid29720743}].



%Genetic abnormalities of the conceptus are a recognized cause of SA and RPL. In literature the prevalence of chromosome abnormalities in this cases was 40-50\%. Determining the chromosomal status of pregnancy tissue from women with recurrent pregnancy loss may provide them with a cause or reason for the particular loss being investigated. No clear effect of genetic testing of the pregnancy tissue on prognosis (subsequent live birth) has been described so far and the role of genetic analysis of pregnancy tissue should be further elaborated within a prognostic model. If women are offered genetic analysis of pregnancy tissue, they should be aware of the issues as mentioned.


%It was decided to recommend parental karyotyping in RPL couples only after an individual risk assessment. Parental karyotyping can be recommended based on genetic history (for instance in case of the previous birth of a child with congenital abnormalities, offspring with unbalanced chromosome abnormalities in the family, or detection of a translocation in the pregnancy tissue). For other couples, the benefit of the test is limited as the chances of finding an abnormality are very low: in couples with female age above 39, less than three pregnancy losses and a negative family history, the chance of being a carrier of a translocation is very low.
%Parental karyotyping may provide couples with a possible contributing factor and prognostic information for the subsequent pregnancy. Regarding prognosis, couples should be informed that, even if a parental abnormality is found after karyotyping, the cumulative live birth rates are good, as are the chances of a healthy child, despite a higher risk of a subsequent pregnancy loss. Furthermore, they should be informed of the limitations of karyotyping, including that karyotyping does not predict unbalanced translocation in next pregnancy.